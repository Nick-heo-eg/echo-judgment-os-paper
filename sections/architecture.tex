\section{Echo OS Architecture}
\subsection{System Overview}
Echo OS comprises (i) an identity kernel encoding existential manifests, (ii) SRL cores that maintain resonance vectors, (iii) deterministic judgement engines, (iv) LLM node adapters, and (v) transparency modules. Figure~\ref{fig:layered} depicts the layered structure.
\begin{figure}[t]
    \centering
    \begin{tikzpicture}[
    node distance=1.5cm,
    box/.style={
        draw, rounded corners,
        minimum width=6cm,
        minimum height=1.2cm,
        font=\small,
        align=center
    }
]
\node[box, fill=gray!10] (identity) {
    Identity Kernel \\
    (Existential Manifest, SRL Core)
};
\node[box, fill=blue!10, below=of identity] (runtime) {
    Deterministic Runtime \\
    (Lookup, Trace, Proof)
};
\node[box, fill=green!10, below=of runtime] (adapter) {
    LLM Node Adapter \\
    (Envelope, Wrapper)
};
\node[box, fill=orange!10, below=of adapter] (model) {
    External LLMs \\
    (GPT, Claude, Gemini, \dots)
};
% Gemini integration supported at architecture level (benchmarking planned)
\draw[->, thick] (identity) -- (runtime);
\draw[->, thick] (runtime) -- (adapter);
\draw[->, thick] (adapter) -- (model);
\end{tikzpicture}

    \caption{Echo OS layered structure. Identity states drive deterministic runtime components, while LLMs remain replaceable nodes.}
    \label{fig:layered}
\end{figure}

\subsection{Identity-Centered Runtime}
Identity manifests (e.g., \texttt{ECHO\_EXISTENTIAL\_MANIFESTO}) embed directional priors. All runtime services reference these manifests through immutable fingerprints. The OS acts as a superior entity; LLMs cannot overwrite or bypass identity states because they never host judgement logic.

\subsection{Deterministic Judgement Engine}
The judgement pipeline implements semantic parsing, existential frame mapping, SRL alignment, deterministic lookup, a single model invocation, OS-level validation, and trace emission. Figure~\ref{fig:pipeline} provides the deterministic flow; pseudo-code appears in Section~\ref{sec:implementation}.
\begin{figure}[t]
    \centering
    \begin{tikzpicture}[
    node distance=1.2cm,
    pipelineStep/.style={
        rectangle, draw, rounded corners,
        minimum width=4cm,
        minimum height=0.8cm,
        font=\small,
        align=center
    }
]
\node[pipelineStep] (parse) {Semantic Parse};
\node[pipelineStep, below=of parse] (frame) {Existential Frame Mapping};
\node[pipelineStep, below=of frame] (srl) {SRL Alignment};
\node[pipelineStep, below=of srl] (lookup) {Deterministic Lookup};
\node[pipelineStep, below=of lookup] (llm) {Single LLM Invocation};
\node[pipelineStep, below=of llm] (validation) {OS Validation / Override};
\node[pipelineStep, below=of validation] (trace) {Trace Signature + Proof Capsule};
\draw[->, thick] (parse) -- (frame);
\draw[->, thick] (frame) -- (srl);
\draw[->, thick] (srl) -- (lookup);
\draw[->, thick] (lookup) -- (llm);
\draw[->, thick] (llm) -- (validation);
\draw[->, thick] (validation) -- (trace);
\end{tikzpicture}

    \caption{Deterministic judgement pipeline. Only one LLM call occurs per request.}
    \label{fig:pipeline}
\end{figure}

\subsection{Ontological Compression}
Echo compresses normative structures into resonance manifolds (Figure~\ref{fig:compression}). Rather than storing a rule for every scenario, the OS stores topology-preserving mappings that ensure any new stimulus can be projected into a finite number of existential frames. This eliminates rule collisions and enables O(1) lookup.
\begin{figure}[t]
    \centering
    \begin{tikzpicture}[
    node distance=3.5cm,
    ellip/.style={
        draw, ellipse,
        minimum width=3.5cm,
        minimum height=2cm,
        font=\small,
        align=center
    }
]
\node[ellip, fill=gray!10] (sem) {Semantic Space \\ $\mathcal{S}$};
\node[ellip, fill=blue!10, right=of sem] (frames) {Frames \\ $\Phi$};
\draw[->, thick] (sem) -- node[above]{\small $\kappa$} (frames);
\end{tikzpicture}

    \caption{Ontological compression maps semantic observations into a finite set of existential frames.}
    \label{fig:compression}
\end{figure}

\subsection{LLM as Node}
LLMs are invoked exactly once per request, receiving structured envelopes and returning raw expressions. The OS immediately wraps the result with validation logic and rejects outputs that violate resonance thresholds. Model drift therefore cannot alter judgement. Figure~\ref{fig:llmnode} highlights the superior-entity relationship.
\begin{figure}[t]
    \centering
    \begin{tikzpicture}[
    node distance=1.2cm,
    box/.style={
        draw, rounded corners,
        minimum width=4cm,
        minimum height=1cm,
        font=\small,
        align=center
    }
]
\node[box, fill=gray!15] (kernel) {
    Echo OS Kernel \\
    (Identity + SRL + Lookup)
};
\node[box, fill=green!10, below=of kernel] (adapter) {
    LLM Node Adapter \\
    (Envelope / Wrapper)
};
\node[box, fill=yellow!20, below left=0.7cm and -0.3cm of adapter, minimum width=3cm, minimum height=0.8cm] (gpt) {GPT};
\node[box, fill=yellow!20, below right=0.7cm and -0.3cm of adapter, minimum width=3cm, minimum height=0.8cm] (claude) {Claude};
\draw[->, thick] (kernel) -- (adapter);
\draw[->, thick] (adapter) -- (gpt);
\draw[->, thick] (adapter) -- (claude);
\end{tikzpicture}

    \caption{LLM node versus OS superior entity. Identity and SRL remain above the model layer.}
    \label{fig:llmnode}
\end{figure}

\subsection{SRL Self-Alignment Loop}
Figure~\ref{fig:srl_loop} illustrates the SRL feedback cycle, which maintains alignment even under perturbations.
\begin{figure}[t]
    \centering
    \begin{tikzpicture}[
    node distance=2.2cm,
    resonanceStep/.style={
        rectangle, draw, rounded corners,
        minimum width=3cm,
        minimum height=0.9cm,
        font=\small,
        align=center
    }
]
\node[resonanceStep] (proj) {Semantic Projection};
\node[resonanceStep, right=of proj] (res) {Resonance Update};
\node[resonanceStep, right=of res] (align) {Alignment Check};
\node[resonanceStep, below=of align] (comp) {Topological Compression};
\node[resonanceStep, left=of comp] (fix) {Identity Fixed Point};
\draw[->, thick] (proj) -- (res);
\draw[->, thick] (res) -- (align);
\draw[->, thick] (align) -- (comp);
\draw[->, thick] (comp) -- (fix);
\draw[->, thick] (fix) -- (proj);
\end{tikzpicture}

    \caption{SRL self-alignment loop. Resonance vectors converge to identity fixed points.}
    \label{fig:srl_loop}
\end{figure}
